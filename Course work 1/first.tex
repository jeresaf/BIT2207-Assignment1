\documentclass{article}
\raggedbottom
\begin{document}

\title{APPROACH TO A SUCCESSFUL DRINK-UP}
\date{01-April-2017}
\author{AYIKO JEREMIAH SARA\\15/U/186\\215001438\\BIT 2207 RESEARCH METHODOLOGY}


\maketitle
\newpage


\begin{abstract}
University students are known to organize drink ups frequently while at hostel. However, many of these events fail due to a lack of organization and structure. The purpose of this document is to define a structure that can be used to host drink-ups successfully. The structure provided gives an idea of how to organize mentally and physically for a drink-up as an organizer. 
\end{abstract}
\newpage

\section{Time and location constraints}
It is important to first consider the location of the drink up, that tells us who is going to be the main host. Once the host is identified, it is important for him/her to know the limitations of hosting:

\begin{itemize}
\item
The host is supposed to stay in the room at least 90\% of the time during the course of the party. This is to handle any unwanted guests i.e. the landlord more often called the custodian.
\item
The host is meant to watch the drunken ravers as they cause chaos in his dwelling. Kick out those meant to be excommunicated and help the falling ladies back to their feet with minimal interference from the drunk.
\end{itemize}
Given a set location, time becomes the next priority, set it too early and only our dear vacationists or first years will appear, set it too late and there will be too much paranoia about how people will make it to their humble abodes. Pick a time based on the schedules of the confirmed guests of your gathering. 
\paragraph{}
With these affairs settled we move to the next constraint.

\section{People}
It is important to note the kind of people who are going to attend the drink-up. The following classification can be used.
\subsection{The high class}
These are the students who drink from bottles or must see at least a liquor bottle or two to participate in the race to finish the alcohol. They are usually defined as the babies i.e. bottle suckers.
\subsection{The suspicious}
These are the students who always ask what is in the punch, which is a drink that is a mixture of different alcohol and usually have rare allergies to some alcohol (or so they say). These are the best people as they always fall prey to the alcohol they were doubting.
\subsection{The drinkers/ loco(crazy)}
These are the ninjas who take anything they see with no prejudice. Sachets, bottles and punch all go down their throats easy. They come already drunk from their daily activities. They are the hype men of our generation. Extremely hard to deal with when both sober and drunk. These are the second best people to have at the drink-up.
\subsection{The Holier than thou}
First of all, these are the worst people to have at the party. It is recommended that you avoid them. They do not partake in neither intoxication nor dance. Be careful when inviting such, a maximum of one person of this caliber is allowed at a time within the host’s club.
\paragraph{}
After classifying your guests, it is best to adjust the ratio of the people attending so that it can work for given budgets as discussed next.

\section{Budget}
A budget is the last and most important consideration. There are two kinds of budgets:
\subsection{The high budget}
This budget is based on a consideration of a high number of high class people. It starts from shs. 200,000 and above. It covers about 6 bottles of liquor based on the place of purchase. Then of course countless sachets for making the punch which can be concocted in all kinds of sinister ways for maximum turn up i.e. almost everybody leaves happy.
\subsection{The loco budget}
 This can start from as low as shs. 60,000 to as much as shs. 150,000 how high figures are not recommended because it changes the kind of people necessary for the party. This budget considers purchase of a multitude of sachets and about 1 or 2 trophy bottles for the last men standing. A loco budget is interesting to work with since it deals with the best kind of party people.
\paragraph{}
After settling the budget, then the rest is in the hands of the hospitality of the party people.

\section{Conclusion}
Drink-ups are the easiest things to organize if based on a structure such as the one given. Keep it simple, invite the right people and have a night never to forget.
		


\end{document}